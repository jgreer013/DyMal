There are numerous perspectives with which to understand program behavior, though many of them either lack interpretability or are unable to be generalized to many types of software. Ghosh \textit{et al.} \cite{ghosh_behavior_1999} utilized neural networks and Elman networks to recognize recurrent features in program execution traces, and while successful and generalizable, the features derived by these networks are often unknowable by humans, leading to their general "black-box" sentiment. Sherwood \textit{et al.} \cite{Sherwood_large_2002} utilized Basic Block Vectors to identify behaviors in large-scale software execution (billions of commands), and clustered their behavior based on this model, but again, interpreting what these behaviors or vectors mean in relation to the specific goals of the program is difficult. 

Bowring \textit{et al.} \cite{Bowring_active_2004} applied Markov models and an active learning approach to multiple executions of programs to identify behaviors among them, utilizing extracted execution statistics to create behavioral profiles for programs. While powerful, the execution statistics lose some amount of information compared to execution traces or the program's static assembly code. Also, depending on the classifier(s) used, any extrapolated features may be difficult to interpret. Not only this, but dynamic analysis is often slow and potentially dangerous to collect the necessary data. Mohaisen \textit{et al.} \cite{Mohaisen_systems_2014} developed a patent to describe a generalized malware analysis system which reflects many of the studies done on analyzing programs and malware. Most tend to take on a similar architecture and are primarily focused on classifier performance rather than gaining insights into understanding program behavior.

There exist many behavioral systems specifically tailored to malware as well. Jacob \textit{et al.} \cite{jacob2008behavioral} introduces the general framework by which one would create an interpretable malware detection system and establishes a taxonomy to discuss malware, something which would prove very useful to follow. Andromaly \cite{shabtai2012andromaly}, Droidmat \cite{wu2012droidmat}, and Crowdroid \cite{burguera2011crowdroid} all introduce behavior-based malware detection systems for the Android operating system, though their primary concerns are all specifically related to the detection of malware as opposed to understanding its behavior or purpose, or how it might be different from non-malicious programs.