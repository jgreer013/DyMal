Classifying program behavior, and in particular in the cybersecurity domain as malicious or benign, is an unsolved problem due to the complexity of programs and the ever-evolving state of malware. Extensive work has been done in this area with a multitude of datasets, including Microsoft's 2015 Kaggle dataset \cite{kaggle}, which contains numerous examples of malware with the end task of classifying each program as belonging to one of nine malware families; however, many of these works do not improve our intrinsic understanding of these programs and what makes them different.

Identifying software characteristics is a well-studied problem, but as software and malware change over time, a complete solution does not and most likely cannot exist without adaptability. This adaptation necessitates developing a better understanding of how programs behave and how they might relate to one another in a way that humans can understand. Even though there are many instances of accurate classifiers, much of the work done is primarily focused on creating new classifiers or comparing them against other models to gauge effectiveness, while very few if any of the works go into understanding the differences between software or the features which are derived by their models. While this paper does not utilize malware classification, which is instead left as future work, it is beneficial to maintain this mindset for the purposes of understanding some of the broader implications of this work and what it brings to both the machine learning and software analysis domains.