This paper details an initial proof of concept on a small-scale sort versus search data set, with additional testing on a large-scale data set. While this study did not focus on malware, its applicability to the malware analysis domain is clear. The difficulty with studying malware stems from safely acquiring copies of malware and using them for study. The most widely available data set is Microsoft's Kaggle data set \cite{kaggle}, which can be used to try to classify malware into one of nine families. This would be the most immediately useful data set to study, though due to its size, performance may become an issue if one does not use an efficient library. The issue with this data set is its lack of benign program samples, which would be necessary if trying to find differentiators between malware and non-malware programs and their corresponding behaviors. One could add the ByteWeight data set to it, but more varied programs and more examples of commercial software would assist in making the approach generalizable.