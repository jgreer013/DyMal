As access to software increases, so too does the risk of downloading or installing malware. Whether it is stealing and selling identification and password data, holding computers or files for ransom from their owners, taking control of a remote system, or simply causing the host software or hardware platform to malfunction, malware comes in a wide variety of forms, and modern software production methodologies have made creating malware even easier, as often times it is simply an alteration of some main source malware \cite{maier2014divide}.

There are many ways to verify that a downloaded program has not been modified from the expected program, such as checking the md5 checksum against the distributor's checksum; however, many of these methods do not guarantee the safety of the program. In addition, programs often may not be doing anything that is particularly malicious, but is instead simply unwanted by the user, as previously mentioned with user data or cryptocurrency mining. Therefore, identifying, classifying, and understanding program behaviors, be they malicious, unwanted, or benign, becomes imperative to ensuring the safety of the user, platform, and program. 

Usually programs which are downloaded from a source or website are generally regarded as trustworthy, such as programs from a secure website or mobile app store; however, there have been instances of trusted software containing malware and being distributed to users, unknown to both the distributor and the users. In 2017, 2.27 million users downloaded a version of CCleaner with malware embedded within, as the assailant was able to modify the program at the distribution site and thus modify the program without the knowledge of CCleaner or the users which downloaded this trusted software\cite{arntz_ccleaner_2017}. In addition, there are numerous instances of malware being downloaded through platforms such as the Google Play Store or Apple App Store \cite{maier2014divide}. As these platforms host millions of apps from developers, it can be difficult to identify potentially malicious software \cite{statista}.